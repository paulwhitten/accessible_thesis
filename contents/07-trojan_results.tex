\chapter{Hardware Trojan Results} \label{ch:trojan_results}

This chapter presents the results of applying the explainable methodologies on
the application of explainable detection of hardware trojans.  The
property-based explainable method suffers from poor explainability while the
case-based explainable method shows promising explainable results by providing
like cases from training along with meaningful metrics showing correspondence to
neighboring training cases and explainability.


\section{Property-Based Explainable Results for Trojan Detection} \label{prop_exp_results_trojan}

SVMs were trained using the three methods: no weighting, static weighting, and
dynamic weighting.  When static weighting used a balance factor, $bf$ from
\ref{eq:balance}, corresponding to the ratio of non-trojans to trojans, static
weighting appeared to perform equivalent to dynamic weighting.

Results of performing unexplainable inference on the test set using the three
methods is shown in Table~\ref{tab_unexp_weight_comp}. Dynamic weighting showed
that it had a true positive rate (TPR) of 0.609, compared to 0.696 for static
weighting, but a slightly lower true negative rate (TNR) at 0.934 versus 0.936
for static weighting. The consequence of this can be observed where dynamic
weighting over reports trojans with 695 false positives (FP) versus 670 FP for
static weighting.

\begin{table}[H]
    \renewcommand{\arraystretch}{1.3}
    %\centering
    \caption{Unexplainable inference results on hardware trojan test data.}
    \begin{center}
    %\resizebox{\columnwidth}{!}
    \end{center}
    \label{tab_unexp_weight_comp}
\end{table}

Results through the property-based explainable architecture with static
weighting resulted in a TPR of 0.696 and a TNR of 0.933. There were 32 true
positives (TP), 8902 true negative (TN), 699 FP, and 14 false negatives (FN) for
static weighting. Results from other weighting methods are depicted in Table
\ref{tab_exp_weight_comp}.  The dynamic method of weighting showed a significant
increase in false positives. In general, the property-based explainable accuracy
results are slightly less than unexplainable results.

\begin{table}[H]
    \renewcommand{\arraystretch}{1.3}
    %\centering
    \caption{Explainable results of inference on the hardware trojan test data using the property-based architecture.}
    \begin{center}
    %\resizebox{\columnwidth}{!}
    \end{center}
    \label{tab_exp_weight_comp}
\end{table}

%1 = sample 1024
%2 = sample 1821
%3 = sample 3237

Table~\ref{tab_ex} depicts the three examples from the test set used to
demonstrate explainability. The first column in the table represents the example
identifier (ID). The next five columns are the features.  The last column
indicates if the sample is a trojan (1) or non-trojan (0). The examples
presented in this section for the property-based architecture are using the
static weighting technique.

\begin{table}[H]
    \renewcommand{\arraystretch}{1.3}
    %\centering
    \caption{Three hardware trojan examples from the test set.}
    \begin{center}
    %\resizebox{\columnwidth}{!}
    \end{center}
    \label{tab_ex}
\end{table}

Hardware trojan example one results, are shown in Table~\ref{tab_prop_exp_ex_1}.
Note the presentation of this table differs from handwritten examples as the
number of properties is about three times the properties used in handwriting.
The rows in the table signify the two possible predictions 1, trojan, and 0,
non-trojan. The confidence column indicates the computed confidence of the
prediction based on the decision-making process using effectiveness metric,
$E_{PARS}$.  The third column shows the property identifiers sorted by
effectiveness over a threshold of 5\%. Explainability is shown in the final
column.

Example one  was appropriately predicted as a trojan (1) with 99.6\% confidence.
Fifteen properties, listed by ID from Table~\ref{tab_prop_id} in order of
effectiveness, contributed sufficiently weighted votes to be registered (above
the 5\% threshold).  Despite the high confidence, explainability was weighted at
41.1\%, because fourteen of the properties above the threshold had three or more
features, so they had limited explainability ($X_j\leq0.5$ in Table
\ref{tab_prop_id}). The alternate decision to identify example one as a trojan
had confidence $<$ 1\% with none of the remaining properties above the
threshold.

\begin{table}[H]
    \renewcommand{\arraystretch}{1.3}
    %\centering
    \caption{Hardware trojan example one property-based explainable results.}
    \begin{center}
    %\resizebox{\columnwidth}{!}{%
    %{|c|c|c|c|}
    \begin{tabular}{|c|c|p{0.32\columnwidth}|c|}
        \hline
         Prediction &  Confidence &  Properties (Table \ref{tab_prop_id}) &  Explainability \\
        \hline
        \hline
        1 & 99.6\% & 31, 28, 26, 29, 27, 30, 21, 18, 16, 19, 24, 17, 22, 25, 15 & 41.1\% \\
        \hline
        0 & 0.4\% & No opinion & 0\% \\
        \hline
    \end{tabular}
    %}
    \end{center}
    \label{tab_prop_exp_ex_1}
\end{table}

The XAI output from the property-based explainable architecture,
Figure~\ref{fig:trojan_arch}, provided the following rationale: ``Trojan was
winner with high confidence due to the property with (LGFi,ffi,ffo,PI,PO)
features, the property with (LGFi,ffi,PI,PO) features, the property with
(LGFi,ffi,ffo,PI) features, the property with (LGFi,ffo,PI,PO) features, the
property with (LGFi,ffi,ffo,PO) features, the property with (ffi,ffo,PI,PO)
features, the property with (LGFi,PI,PO) features, the property with
(LGFi,ffi,PO) features, the property with (LGFi,ffi,ffo) features, the property
with (LGFi,ffo,PI) features, the property with (ffi,PI,PO) features, the
property with (LGFi,ffi,PI) features, the property with (ffi,ffo,PI) features,
the property with (ffo,PI,PO) features, and the property with (PI,PO)
features.''

\begin{table}[H]
    \renewcommand{\arraystretch}{1.3}
    %\centering
    \caption{Hardware trojan example two property-based explainable results.}
    \begin{center}
    %\resizebox{\columnwidth}{!}{%
    \begin{tabular}{|c|c|p{0.32\columnwidth}|c|}
        \hline
         Prediction &  Confidence &  Properties (Table \ref{tab_prop_id}) &  Explainability \\
        \hline
        \hline
        1 & 25.7\% & 30, 24, 22, 25, 15 & 50.0\% \\
        \hline
        0 & 74.3\% & 31, 28, 26, 29, 27, 21, 18, 16, 19, 17 & 35.0\% \\
        \hline
    \end{tabular}
    %}
    \end{center}
    \label{tab_prop_exp_ex_2}
\end{table}

The property-based architecture improperly identified example two, suggesting
the winning decision with medium, 74.3\%, confidence was non-trojan.
Table~\ref{tab_prop_exp_ex_2} illustrates the properties, confidence, and
explainability for example two.  Ten properties voted for a non-trojan, while
five suggested a trojan.  Again, the explainability was low at only 35.0\% for
the winning decision due to the relatively large number of features in voting
properties.

The XAI output rationale provided by the
architecture for decision was cited as: ``Not a trojan was winner with medium
confidence due to the property with (LGFi,ffi,ffo,PI,PO) features, the property
with (LGFi,ffi,PI,PO) features, the property with (LGFi,ffi,ffo,PI) features,
the property with (LGFi,ffo,PI,PO) features, the property with (LGFi,ffi,ffo,PO)
features, the property with (LGFi,PI,PO) features, the property with
(LGFi,ffi,PO) features, the property with (LGFi,ffi,ffo) features, the property
with (LGFi,ffo,PI) features, and the property with (LGFi,ffi,PI) features.'' The
rationale had an average explainability of 0.350 since all ten properties had
three or more features.

Alternatively, for example two with 25.6\% confidence, a trojan was identified
with XAI output rationale: ``Trojan was suggested with medium confidence due to
the property with (ffi,ffo,PI,PO) features, the property with (ffi,PI,PO)
features, the property with (ffi,ffo,PI) features, the property with (ffo,PI,PO)
features, and the property with (PI,PO) features.''  This rationale had an average
explainability 0f 0.500 with only one property having greater than three features.

\begin{table}[H]
    \renewcommand{\arraystretch}{1.3}
    %\centering
    \caption{Hardware trojan example three property-based explainable results.}
    \begin{center}
    %\resizebox{\columnwidth}{!}{%
    \begin{tabular}{|c|c|p{0.32\columnwidth}|c|}
        \hline
         Prediction &  Confidence &  Properties (Table \ref{tab_prop_id}) &  Explainability \\
        \hline
        \hline
        1 & 58.4\% & 28, 26, 21, 18, 16, 20, 24, 22, 15 & 47.2\% \\
        \hline
        0 & 41.6\% & 31, 29, 27, 30, 17, 24 & 29.1\% \\
        \hline
    \end{tabular}
    %}
    \end{center}
    \label{tab_prop_exp_ex_3}
\end{table}

Example three is labeled as non-trojan in the test set.  The architecture
improperly identified the sample as a trojan with medium, 58.4\%, confidence. In
Table~\ref{tab_prop_exp_ex_3} nine properties suggested the sample as a trojan while
six suggested a normal net.  Explainability was among the highest at 47.2\%.

The XAI rationale
was: ``Trojan was winner with medium confidence due to the property with
(LGFi,ffi,PI,PO) features, the property with (LGFi,ffi,ffo,PI) features, the
property with (LGFi,PI,PO) features, the property with (LGFi,ffi,PO) features,
the property with (LGFi,ffi,ffo) features, the property with (LGFi,ffo,PI)
features, the property with (ffi,PI,PO) features, the property with (ffi,ffo,PI)
features, and the property with (PI,PO) features.'' The average explainability
of this rationale was 0.472.

In example three, six properties suggested the sample was not a trojan with a
confidence of 41.6\%. The XAI rationale provided was ``Not a trojan was
suggested with medium confidence due to the property with (LGFi,ffi,ffo,PI,PO)
features, the property with (LGFi,ffo,PI,PO) features, the property with
(LGFi,ffi,ffo,PO) features, the property with (ffi,ffo,PI,PO) features, the
property with (LGFi,ffi,PI) features, and the property with (ffo,PI,PO)
features.''  Average explainability for this rationale was 0.291.

\section{Case-Based Explainable Results for Trojan Detection} \label{case_exp_results}

The case based explainable architecture performed well, providing relevant KNN
for all of the test samples.  Weights of KNN corresponded to the decision of the
architecture in 97.4\% of the cases in the 20\% test set.

The SVM in the case-based architecture for hardware trojan example one voted for
a trojan.  The KNN cases provided by the architecture are shown in
Table~\ref{tab_index_exp_ex_1}.  In the table nearest neighbor groups are shown
in rows. The distance column indicates the Euclidean distance to the sample
group.  A distance of zero indicates a sample group has the same feature vector
as the sample.  The feature values indicate the feature vector of the training
sample(s).  The $t:n$ column indicates the ratio of trojan ($t$) to non-trojan
($n$) nets for the samples in the group at the point in feature space. The
$WN(class)$ columns indicate the weight of neighbors for each class, trojan
($t$) and non-trojan ($n$) from \ref{eq:weight_of_neighbors}.

%1 = sample 1024

\begin{table}[H]
    \renewcommand{\arraystretch}{1.3}
    %\centering
    \caption{Hardware trojan example one case-based explainable results.}
    \begin{center}
    %\resizebox{\columnwidth}{!}{%
    \begin{tabular}{|c|c|c|r|r|}
        \hline
         Distance &  Feature Values & $t:n$ & \multicolumn{1}{c|}{$WN(t)$} & \multicolumn{1}{c|}{$WN(n)$} \\
        \hline
        \hline
        0.00 & \textlangle 8, 1, 3, 2, 3\textrangle & 11:0 & 2893.0 & 0.00 \\
        \hline
        1.00 & \textlangle 8, 2, 3, 2, 3\textrangle & 0:34 & 0.00 & 8.50 \\
        \hline
        1.00 & \textlangle 8, 1, 3, 1, 3\textrangle & 1:204 & 65.75 & 51.00 \\
        \hline
        1.41 & \textlangle 8, 1, 4, 2, 4\textrangle & 3:0 & 135.37 & 0.00 \\
        \hline
        %\cline{2-5}
        \multicolumn{1}{c|}{} & \multicolumn{1}{l|}{Sum} & $15+238=253$ & 3094.12 & 59.50 \\
        \cline{2-5}
        \multicolumn{2}{c|}{} & \multicolumn{1}{l|}{Correspondence} & 98.1\% & 1.9\% \\
        \cline{3-5}
    \end{tabular}
    %}
    \end{center}
    \label{tab_index_exp_ex_1}
\end{table}

For example, the first row of Table~\ref{tab_index_exp_ex_1} indicates there are
eleven training samples grouped as the same point as hardware trojan example
one. all of the samples in the group were trojans.  The resulting weight of
trojans from that grouping is $2893.0$.  The second row indicates that at a
distance of $1.0$ from example one at $\textlangle 8, 2, 3, 2, 3\textrangle$
there is a group of 34 non-trojans.  The non-trojan weight for that group is
$8.50$.  The third row shows also  at a distance of $1.0$ from example one there
is a grouping at $\textlangle 8, 1, 3, 1, 3\textrangle$ with one trojan and 204
non-trojans.  Because of the balance factor, the trojan weight is $65.75$ while
the non-trojan weight is $51.00$.  The final row indicates a group of three
trojans are found a distance of $1.41$ from example one. The trojan weight of
this group is $135.37$. Summing the wight rows results in an overall trojan
weight of $3094.12$ to non-trojan $59.50$.

%TODO add wi columns and explainability
%add test to indicate 253 is the sum of column Trojan:Normal

The case-based architecture further provides a quantitative measure for hardware
trojan example one indicating the decision corresponds with neighbor weighting
by suggesting a correspondence of 98.1\% for the prediction of trojan based on
neighbors from training data.

References to the neighboring cases for example one are helpful in identifying
those cases in context.  Neighbors at $\textlangle 8, 1, 3, 2, 3\textrangle$
include all eleven matches e.g., part number RS232, version T1500, line 40, name
NAND4X1, and net U294.QN as well as part RS232, version T1000, line 35, name
NAND4X1, and net U299.QN. The output of the explainability results produces all
references to the 253 KNN examples from the case-based training set.

%2 = sample 1821

\begin{table}[H]
    \renewcommand{\arraystretch}{1.3}
    %\centering
    \caption{Hardware trojan example two case-based explainable results}
    \begin{center}
    %\resizebox{\columnwidth}{!}{%
    \begin{tabular}{|c|c|c|r|r|}
        \hline
         Distance &  Feature Values & $t:n$ & \multicolumn{1}{c|}{$WN(t)$} & \multicolumn{1}{c|}{$WN(n)$} \\
        \hline
        \hline
        0.00 & \textlangle 3, 3, 5, 3, 5\textrangle & 0:57 & 0.00 & 57.00 \\
        \hline
        1.00 & \textlangle 4, 3, 5, 3, 5\textrangle & 1:24 & 65.75 & 6.00 \\
        \hline
        1.00 & \textlangle 2, 3, 5, 3, 5\textrangle & 0:104 & 0.00 & 26.00 \\
        \hline
        2.00 & \textlangle 5, 3, 5, 3, 5\textrangle & 0:11 & 0.00 & 1.22 \\
        \hline
        %\cline{2-5}
        \multicolumn{1}{c|}{} & \multicolumn{1}{l|}{Sum} & $1+196=197$ & 65.75 & 90.22 \\
        \cline{2-5}
        \multicolumn{2}{c|}{} & \multicolumn{1}{l|}{Correspondence} & 42.2\% & 57.8\% \\
        \cline{3-5}
    \end{tabular}
    %}
    \end{center}
    \label{tab_index_exp_ex_2}
\end{table}

Example two was incorrectly identified as non-trojan by the case-based
architecture. The KNN cases are depicted in Table~\ref{tab_index_exp_ex_2}. The
architecture assigned a medium correspondence of 57.8\% for the prediction of no
trojan based on neighbors from training data. This was due to the single
neighboring trojan's weight at distance one being overcome by other neighboring
196 samples. References to all 197 samples from training were provided for
context.

%3 = sample 3237

\begin{table}[H]
    \renewcommand{\arraystretch}{1.3}
    %\centering
    \caption{Hardware trojan example three case-based explainable results.}
    \begin{center}
    %\resizebox{\columnwidth}{!}{%
    \begin{tabular}{|c|c|c|r|r|}
        \hline
         Distance &  Feature Values & $t:n$ & \multicolumn{1}{c|}{$WN(t)$} & \multicolumn{1}{c|}{$WN(n)$} \\
        \hline
        \hline
        0.00 & \textlangle 5, 2, 14, 2, 14\textrangle & 0:1 & 0.00 & 1.00 \\
        \hline
        1.41 & \textlangle 5, 2, 13, 2, 13\textrangle & 0:2 & 0.00 & 0.34 \\
        \hline
        1.73 & \textlangle 4, 2, 13, 2, 13\textrangle & 7:0 & 246.65 & 0.00 \\
        \hline
        2.45 & \textlangle 3, 2, 13, 2, 13\textrangle & 3:0 & 66.31 & 0.0 \\
        \hline
        %\cline{2-5}
        \multicolumn{1}{c|}{} & \multicolumn{1}{l|}{Sum} & $10+3=13$ & 312.96 & 1.34 \\
        \cline{2-5}
        \multicolumn{2}{c|}{} & \multicolumn{1}{l|}{Correspondence} & 99.6\% & 0.4\% \\
        \cline{3-5}
    \end{tabular}
    %}
    \end{center}
    \label{tab_index_exp_ex_3}
\end{table}

Example three, shown in Table~\ref{tab_index_exp_ex_3}, was correctly identified
as not a trojan by the SVM.  The KNN gave a rare conflicting 99.6\%
correspondence metric for a trojan contradicting the prediction of the SVM.  The
two closest samples from training were all non-trojan with three samples.  The
three samples were overcome by the ten samples that were the next closest due to
the balance.


In example three, the case-based explainable architecture further went on to
explain, ``The decision contradicts the high correspondence $99.6\%$ for the
prediction of a trojan based on neighbors from training data.''  This indicates
the incidence of the ten trojans, despite their further distance, outweighed the
three closer non-trojan training samples.  Such disagreement could be approached
by modifying the NN weighting to decrease the influence based on distance.  A
scheme in \ref{eq:weight_of_neighbors}, taking the inverse cube of distance
could help to lower the impact of trojans further from the considered sample.
Like all other samples, all originating data from the thirteen KNN training
samples were provided for context by the case-based explainable architecture. 

