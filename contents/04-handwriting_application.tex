\chapter{Application to Handwritten Characters} \label{ch:handwriting}
% remove below this line and add your method(s)

This chapter first outlines the application of the explainable property-based
methodology and then the case-based approach to detect handwritten characters
explainably. The property-based approach utilized both the MNIST and EMNIST
datasets, while the case-based approach only utilized MNIST. The
procedures are detailed and discussed when implementing the methodologies as
they apply to the handwritten datasets.

\section{The Explainable Property-Based Methodology and MNIST}

This section will apply the explainable property-based methodology to the
problem of explainably detecting handwritten digits in the MNIST dataset.
Following the property-based methodology, Python programs were developed to
implement the components of the explainable architecture.

\subsection{Data Processing}

The MNIST dataset required no specialized data processing as it has been used in
much AI research. Python functions were written to deserialize the database for
use by the various machine learning methods. The implementation of those
functions is in the file
\href{https://gitlab.com/paulwhitten/2023-svm-xai/-/blob/main/load_mnist_data.py}{\texttt{load\_mnist\_data.py}}\cite{whitten2024icmi_source}.

\subsection{Explainable Properties}

During the discovery of explainable properties, significant time was spent
combing through MNIST examples, discussing and posing geometric features of
digits. An enumeration of the explainable properties posed follows:

\begin{itemize}
    \item Area - The area of a digit is the number of pixels activated.
    \item Circles - A circle is a full or partially-rounded region.
    \item Corners - A corner is an inflection point.
    \item Crossing - A crossing is where two lines (strokes) meet.
    \item Curves - A curve is a line that deviates from straight without a sharp inflection point.
    \item Enclosed regions - An enclosed region is a fully enclosed set of pixels.
    \item Lines - Lines are two points connected by a minimal extent.
    \item Slant - A slant, slope, or lean in the direction of a digit.
    \item Stroke - The stroke of a digit is the minimal outline of the path of the writing implement.
\end{itemize}

\subsection{Property Transforms}

Following the explainable property-based methodology, transforms of the MNIST
input images related to explainable properties are sought. Some properties, such
as the area, curves, and slants, were discarded due to the lack of utility in
differentiating between classes (area) or problems with identifying and applying
appropriate transforms (curve and slant).

Digital image processing techniques that highlight the explainable properties
are used as transforms. Table~\ref{tab:transsample} shows transforms associated
with the remaining properties. The first column is the property transform $T$
and its identifier $j$. The second column indicates the property associated with
the transform. The transform column describes the transformation algorithm used.
The final column provides an example transformation for application input,
$I_i$, an MNIST digit. To the right, the transformed MNIST digit is shown as
$T_j(I_i)$.

\renewcommand{\arraystretch}{1.8}
%\setlength\tabcolsep{2mm}
\begin{table}[h]
\renewcommand{\arraystretch}{1.3}
\caption{Properties and transforms for the MNIST example}
\begin{center}
%\resizebox{\columnwidth}{!}{%
\begin{tabular}{ c | c | c | ccc | }
\cline{2-6}
$T_j$ & Property & Transform & $I_i$ &  &  $T_j(I_i)$ \\
\hline \hline
$T_1$ & Stroke & Skeleton & \raisebox{-.5\height}{\includegraphics[width=5.5mm, alt={An MNIST digit}]{./images/digit-images/4-11.png}} & $\rightarrow$ & \raisebox{-.5\height}{\includegraphics[width=5.5mm, alt={A transformed MNIST digit}]{./images/digit-images/4-11-skel.png}} \\
\hline
$T_2$ & Circle & Hough Circle & \raisebox{-.5\height}{\includegraphics[width=5.5mm, alt={An MNIST digit}]{./images/digit-images/6-17.png}} & $\rightarrow$ & \raisebox{-.5\height}{\includegraphics[width=5.5mm, alt={A transformed MNIST digit}]{./images/digit-images/6-17-circle.png}} \\
\hline
$T_3$ & Crossings & Crossings & \raisebox{-.5\height}{\includegraphics[width=5.5mm, alt={An MNIST digit}]{./images/digit-images/4-2.png}} & $\rightarrow$ & \raisebox{-.5\height}{\includegraphics[width=5.5mm, alt={A transformed MNIST digit}]{./images/digit-images/4-2-crossing.png}} \\
\hline
$T_4$ & Circle & Hough Ellipse & \raisebox{-.5\height}{\includegraphics[width=5.5mm, alt={An MNIST digit}]{./images/digit-images/0-3.png}} & $\rightarrow$ & \raisebox{-.5\height}{\includegraphics[width=5.5mm, alt={A transformed MNIST digit}]{./images/digit-images/0-3-ellipse.png}} \\
\hline
$T_5$ & Circle & Multiple Ellipse Circle & \raisebox{-.5\height}{\includegraphics[width=5.5mm, alt={An MNIST digit}]{./images/digit-images/8-4.png}} & $\rightarrow$ & \raisebox{-.5\height}{\includegraphics[width=5.5mm, alt={A transformed MNIST digit}]{./images/digit-images/8-4-ellipse-circle.png}} \\
\hline
$T_6$ & Endpoints & Endpoints & \raisebox{-.5\height}{\includegraphics[width=5.5mm, alt={An MNIST digit}]{./images/digit-images/2-2.png}} & $\rightarrow$ & \raisebox{-.5\height}{\includegraphics[width=5.5mm, alt={A transformed MNIST digit}]{./images/digit-images/2-2-endpoint.png}} \\
\hline
$T_7$ & Enclosed Region & Flood Fill & \raisebox{-.5\height}{\includegraphics[width=5.5mm, alt={An MNIST digit}]{./images/digit-images/0-2.png}} & $\rightarrow$ & \raisebox{-.5\height}{\includegraphics[width=5.5mm, alt={A transformed MNIST digit}]{./images/digit-images/0-2-fill.png}} \\
\hline
$T_8$ & Line & Hough Line & \raisebox{-.5\height}{\includegraphics[width=5.5mm, alt={An MNIST digit}, alt={An mnist digit}]{./images/digit-images/7-20.png}} & $\rightarrow$ & \raisebox{-.5\height}{\includegraphics[width=5.5mm, alt={A transformed MNIST digit}]{./images/digit-images/7-20-line.png}} \\
%\hline
%$P_9$ & Enclosed Region & Skeleton Flood Fill & \raisebox{-.5\height}{\includegraphics[width=5.5mm, alt={An MNIST digit}]{./images/digit-images/8-3.png}} & $\rightarrow$ & \raisebox{-.5\height}{\includegraphics[width=5.5mm, alt={A transformed MNIST digit}]{./images/digit-images/8-3-skel-fill.png}} \\
\hline
$T_{9}$ & Enclosed Region & Convex Hull & \raisebox{-.5\height}{\includegraphics[width=5.5mm, alt={An MNIST digit}]{./images/digit-images/0-0-12.png}} & $\rightarrow$ & \raisebox{-.5\height}{\includegraphics[width=5.5mm, alt={A transformed MNIST digit}]{./images/digit-images/0-0-12-ch.png}} \\
\hline
$T_{10}$ & Corner & Harris Corner & \raisebox{-.5\height}{\includegraphics[width=5.5mm, alt={An MNIST digit}]{./images/digit-images/4-26.png}} & $\rightarrow$ & \raisebox{-.5\height}{\includegraphics[width=5.5mm, alt={A transformed MNIST digit}]{./images/digit-images/4-26-corner.png}} \\
\hline
\end{tabular}%
%}
\end{center}
\centering
\label{tab:transsample}
\end{table}

Most of the transform algorithms in Table~\ref{tab:transsample} used
implementations in the Python package scikit-image\cite{scikitimage}. A
description of the parameters used follows. Particulars of the transformations
can also be found in the source code repo in the file
\href{https://gitlab.com/paulwhitten/2023-svm-xai/-/blob/main/transform_parallel.py}{\texttt{transform\_parallel.py}}\cite{whitten2024icmi_source}.

The Stroke property transform, $T_1$, represents the minimal path of the writing
implement to trace the digit. The morphological skeleton transformation is a
one-pixel connected representation of the digit, highlighting the stroke. The
Lee\cite{LEE1994} algorithm for the skeleton is used. Figure
\ref{fig:skeleton_examples} depicts sample transforms of digits. The original
digit is on the left, and the transform is shown on the right. The transforms
from \ref{fig:stroke_orig1} $\rightarrow$ \ref{fig:stroke_transform1} and
\ref{fig:stroke_orig3} $\rightarrow$ \ref{fig:stroke_transform3} are expected,
while the transform from \ref{fig:stroke_orig2} $\rightarrow$
\ref{fig:stroke_transform2} seems to lose some detail and resembles a digit two.

\begin{figure}[H]
    \centering

    \begin{subfigure}{.15\columnwidth}
        \centering
        \includegraphics[width=.90\textwidth, alt={An MNIST digit}]{./images/mnist_transforms/raw/1-0.png}
        \caption{}
        \label{fig:stroke_orig1}
    \end{subfigure}%
    $\rightarrow$
    \begin{subfigure}{.15\columnwidth}
        \centering
        \includegraphics[width=.90\textwidth, alt={An MNIST digit}]{./images/mnist_transforms/stroke/1-0.png}
        \caption{}
        \label{fig:stroke_transform1}
    \end{subfigure}

    \par\medskip

    \begin{subfigure}{.15\columnwidth}
        \centering
        \includegraphics[width=.90\textwidth, alt={An MNIST digit}]{./images/mnist_transforms/raw/1-9.png}
        \caption{}
        \label{fig:stroke_orig2}
    \end{subfigure}%
    $\rightarrow$
    \begin{subfigure}{.15\columnwidth}
        \centering
        \includegraphics[width=.90\textwidth, alt={An MNIST digit}]{./images/mnist_transforms/stroke/1-9.png}
        \caption{}
        \label{fig:stroke_transform2}
    \end{subfigure}%

    \par\medskip

    \begin{subfigure}{.15\columnwidth}
        \centering
        \includegraphics[width=.90\textwidth, alt={An MNIST digit}]{./images/mnist_transforms/raw/8-18.png}
        \caption{}
        \label{fig:stroke_orig3}
    \end{subfigure}%
    $\rightarrow$
    \begin{subfigure}{.15\columnwidth}
        \centering
        \includegraphics[width=.90\textwidth, alt={An MNIST digit}]{./images/mnist_transforms/stroke/8-18.png}
        \caption{}
        \label{fig:stroke_transform3}
    \end{subfigure}%

    \caption{Examples of the stroke of digits by taking the skeleton.}
    \label{fig:skeleton_examples}
\end{figure}

Note that three property transforms are associated with the circle property as
$T_2$, $T_4$, and $T_5$. The circle property saw challenges in identifying
various circular structures in MNIST digits. Some of the transforms related to
the circle property were specialized for particular digits. Digits like the zero
often had an elongated circle, which was not as readily detected using the Hough
Circle. Other digits, like the eight, often had two circular features that
should both be detected. It was observed that Hough circles are effective for
detecting digits six and nine but generally perform poorly for digits eight and
zero.

The ellipse transform is better at extracting characteristics of handwritten
zeros. Detection of the two circular features in most digit eights was improved
using a transform for multiple non-overlapping circles or an ellipse.

$T_2$ involves the individual circle property with the Hough Circle transform.
The Hough circle detector used detection on a radius range from three to
thirteen pixels. This transformation was more effective in identifying digits
like six and nine. Figure \ref{fig:circle_examples} illustrates some circle
transforms. The first two, showing some of the challenges with zero in the
transform \ref{fig:circle_orig1} $\rightarrow$ \ref{fig:circle_transform1} and
\ref{fig:circle_orig2} $\rightarrow$ \ref{fig:circle_transform2}, show how
irregularities in the digit could result in the detection of small circles. The
transform of the eight in \ref{fig:circle_orig4} $\rightarrow$
\ref{fig:circle_transform4} depicts the trouble detecting both circular features
of the eight. The digits six and nine depicted in \ref{fig:circle_orig3} $\rightarrow$
\ref{fig:circle_transform3} and \ref{fig:circle_orig5} $\rightarrow$
\ref{fig:circle_transform5} are representative of the Hough circle on those
digits.


\begin{figure}[H]
    \centering

    \begin{subfigure}{.15\columnwidth}
        \centering
        \includegraphics[width=.90\textwidth, alt={An MNIST digit}]{./images/mnist_transforms/raw/0-28.png}
        \caption{}
        \label{fig:circle_orig1}
    \end{subfigure}%
    $\rightarrow$
    \begin{subfigure}{.15\columnwidth}
        \centering
        \includegraphics[width=.90\textwidth, alt={An MNIST digit}]{./images/mnist_transforms/circle/0-28.png}
        \caption{}
        \label{fig:circle_transform1}
    \end{subfigure}

    \par\medskip

    \begin{subfigure}{.15\columnwidth}
        \centering
        \includegraphics[width=.90\textwidth, alt={An MNIST digit}]{./images/mnist_transforms/raw/0-19.png}
        \caption{}
        \label{fig:circle_orig2}
    \end{subfigure}%
    $\rightarrow$
    \begin{subfigure}{.15\columnwidth}
        \centering
        \includegraphics[width=.90\textwidth, alt={An MNIST digit}]{./images/mnist_transforms/circle/0-19.png}
        \caption{}
        \label{fig:circle_transform2}
    \end{subfigure}%

    \par\medskip

    \begin{subfigure}{.15\columnwidth}
        \centering
        \includegraphics[width=.90\textwidth, alt={An MNIST digit}]{./images/mnist_transforms/raw/6-18.png}
        \caption{}
        \label{fig:circle_orig3}
    \end{subfigure}%
    $\rightarrow$
    \begin{subfigure}{.15\columnwidth}
        \centering
        \includegraphics[width=.90\textwidth, alt={An MNIST digit}]{./images/mnist_transforms/circle/6-18.png}
        \caption{}
        \label{fig:circle_transform3}
    \end{subfigure}%

    \par\medskip

    \begin{subfigure}{.15\columnwidth}
        \centering
        \includegraphics[width=.90\textwidth, alt={An MNIST digit}]{./images/mnist_transforms/raw/8-20.png}
        \caption{}
        \label{fig:circle_orig4}
    \end{subfigure}%
    $\rightarrow$
    \begin{subfigure}{.15\columnwidth}
        \centering
        \includegraphics[width=.90\textwidth, alt={An MNIST digit}]{./images/mnist_transforms/circle/8-20.png}
        \caption{}
        \label{fig:circle_transform4}
    \end{subfigure}%

    \par\medskip

    \begin{subfigure}{.15\columnwidth}
        \centering
        \includegraphics[width=.90\textwidth, alt={An MNIST digit}]{./images/mnist_transforms/raw/9-16.png}
        \caption{}
        \label{fig:circle_orig5}
    \end{subfigure}%
    $\rightarrow$
    \begin{subfigure}{.15\columnwidth}
        \centering
        \includegraphics[width=.90\textwidth, alt={An MNIST digit}]{./images/mnist_transforms/circle/9-16.png}
        \caption{}
        \label{fig:circle_transform5}
    \end{subfigure}%

    \caption{Examples of circles in digits by taking the Hough circle.}
    \label{fig:circle_examples}
\end{figure}

The Hough Ellipse was used in $T_4$ for detecting the zero. Detection of
ellipses used a pixel range between five and twenty-three pixels. Figure
\ref{fig:ellipse_examples} shows the exact zeros from Figure
\ref{fig:circle_examples} that the Hough circle had difficulty detecting. Also,
in Figure \ref{fig:circle_examples}, the transform of an eight is shown where
the overall shape is detected as a single large ellipse.

\begin{figure}[H]
    \centering

    \begin{subfigure}{.15\columnwidth}
        \centering
        \includegraphics[width=.90\textwidth, alt={An MNIST digit}]{./images/mnist_transforms/raw/0-28.png}
        \caption{}
        \label{fig:ellipse_orig1}
    \end{subfigure}%
    $\rightarrow$
    \begin{subfigure}{.15\columnwidth}
        \centering
        \includegraphics[width=.90\textwidth, alt={An MNIST digit}]{./images/mnist_transforms/ellipse/0-28.png}
        \caption{}
        \label{fig:ellipse_transform1}
    \end{subfigure}

    \par\medskip

    \begin{subfigure}{.15\columnwidth}
        \centering
        \includegraphics[width=.90\textwidth, alt={An MNIST digit}]{./images/mnist_transforms/raw/0-19.png}
        \caption{}
        \label{fig:ellipse_orig2}
    \end{subfigure}%
    $\rightarrow$
    \begin{subfigure}{.15\columnwidth}
        \centering
        \includegraphics[width=.90\textwidth, alt={An MNIST digit}]{./images/mnist_transforms/ellipse/0-19.png}
        \caption{}
        \label{fig:ellipse_transform2}
    \end{subfigure}%

    \par\medskip

    \begin{subfigure}{.15\columnwidth}
        \centering
        \includegraphics[width=.90\textwidth, alt={An MNIST digit}]{./images/mnist_transforms/raw/8-20.png}
        \caption{}
        \label{fig:ellipse_orig3}
    \end{subfigure}%
    $\rightarrow$
    \begin{subfigure}{.15\columnwidth}
        \centering
        \includegraphics[width=.90\textwidth, alt={An MNIST digit}]{./images/mnist_transforms/ellipse/8-20.png}
        \caption{}
        \label{fig:ellipse_transform3}
    \end{subfigure}

    \caption{Examples of the ellipse in digits by taking the Hough ellipse.}
    \label{fig:ellipse_examples}
\end{figure}

The transform for multiple non-overlapping circles and ellipse implemented in
$T_5$ effectively detected the digit eight as two circular features as shown in
Figure \ref{fig:ellipse_circle_examples}.

\begin{figure}[H]
    \centering

    \begin{subfigure}{.15\columnwidth}
        \centering
        \includegraphics[width=.90\textwidth, alt={An MNIST digit}]{./images/mnist_transforms/raw/8-20.png}
        \caption{}
        \label{fig:ellipse_circle_orig1}
    \end{subfigure}%
    $\rightarrow$
    \begin{subfigure}{.15\columnwidth}
        \centering
        \includegraphics[width=.90\textwidth, alt={An MNIST digit}]{./images/mnist_transforms/ellipse_circle/8-20.png}
        \caption{}
        \label{fig:ellipse_circle_transform1}
    \end{subfigure}

    \caption{Examples of the multiple non-overlapping circles and ellipse transform.}
    \label{fig:ellipse_circle_examples}
\end{figure}


The crossings property transform, $T_3$, represents the intersection of segments
in a digit. The transform involves taking the skeleton and finding activated
pixels with more than two neighbors, indicating a confluence of segments. The
crossing transform had mixed utility as many digits provide no pixels activated
in the output, and other digits provide ambiguous output as shown in Figure
\ref{fig:crossing_examples}.


\begin{figure}[H]
    \centering

    \begin{subfigure}{.15\columnwidth}
        \centering
        \includegraphics[width=.90\textwidth, alt={An MNIST digit}]{./images/mnist_transforms/raw/3-3.png}
        \caption{}
        \label{fig:crossing_orig1}
    \end{subfigure}%
    $\rightarrow$
    \begin{subfigure}{.15\columnwidth}
        \centering
        \includegraphics[width=.90\textwidth, alt={An MNIST digit}]{./images/mnist_transforms/crossing/3-3.png}
        \caption{}
        \label{fig:crossing_transform1}
    \end{subfigure}

    \par\medskip

    \begin{subfigure}{.15\columnwidth}
        \centering
        \includegraphics[width=.90\textwidth, alt={An MNIST digit}]{./images/mnist_transforms/raw/4-5.png}
        \caption{}
        \label{fig:crossing_orig2}
    \end{subfigure}%
    $\rightarrow$
    \begin{subfigure}{.15\columnwidth}
        \centering
        \includegraphics[width=.90\textwidth, alt={An MNIST digit}]{./images/mnist_transforms/crossing/4-5.png}
        \caption{}
        \label{fig:crossing_transform2}
    \end{subfigure}

    \par\medskip

    \begin{subfigure}{.15\columnwidth}
        \centering
        \includegraphics[width=.90\textwidth, alt={An MNIST digit}]{./images/mnist_transforms/raw/6-5.png}
        \caption{}
        \label{fig:crossing_orig3}
    \end{subfigure}%
    $\rightarrow$
    \begin{subfigure}{.15\columnwidth}
        \centering
        \includegraphics[width=.90\textwidth, alt={An MNIST digit}]{./images/mnist_transforms/crossing/6-5.png}
        \caption{}
        \label{fig:crossing_transform3}
    \end{subfigure}

    \caption{Examples of the crossing transform.}
    \label{fig:crossing_examples}
\end{figure}

The endpoints property transform, $T_6$, also involves first taking the skeleton
of the digit. Endpoints are identified as activated pixels in the skeleton with
only one neighbor, which indicates they terminate a segment. Figure
\ref{fig:endpoint_examples} shows endpoint transformations. Figure
\ref{fig:endpoint_orig1} $\rightarrow$ \ref{fig:endpoint_transform1} illustrates
the behavior when a digit, like the zero, has no endpoint. Transform
\ref{fig:endpoint_orig2} $\rightarrow$ \ref{fig:endpoint_transform2} shows the
endpoints on the leftmost tips of the three. The transform of the eight in
\ref{fig:endpoint_orig3} $\rightarrow$ \ref{fig:endpoint_transform3} illustrates
the result of a break in a digit compared to an eight in
\ref{fig:endpoint_orig4} $\rightarrow$ \ref{fig:endpoint_transform4} without a
break.

\begin{figure}[H]
    \centering

    \begin{subfigure}{.15\columnwidth}
        \centering
        \includegraphics[width=.90\textwidth, alt={An MNIST digit}]{./images/mnist_transforms/raw/0-28.png}
        \caption{}
        \label{fig:endpoint_orig1}
    \end{subfigure}%
    $\rightarrow$
    \begin{subfigure}{.15\columnwidth}
        \centering
        \includegraphics[width=.90\textwidth, alt={An MNIST digit}]{./images/mnist_transforms/endpoint/0-28.png}
        \caption{}
        \label{fig:endpoint_transform1}
    \end{subfigure}

    \par\medskip

    \begin{subfigure}{.15\columnwidth}
        \centering
        \includegraphics[width=.90\textwidth, alt={An MNIST digit}]{./images/mnist_transforms/raw/3-0.png}
        \caption{}
        \label{fig:endpoint_orig2}
    \end{subfigure}%
    $\rightarrow$
    \begin{subfigure}{.15\columnwidth}
        \centering
        \includegraphics[width=.90\textwidth, alt={An MNIST digit}]{./images/mnist_transforms/endpoint/3-0.png}
        \caption{}
        \label{fig:endpoint_transform2}
    \end{subfigure}%

    \par\medskip

    \begin{subfigure}{.15\columnwidth}
        \centering
        \includegraphics[width=.90\textwidth, alt={An MNIST digit}]{./images/mnist_transforms/raw/8-20.png}
        \caption{}
        \label{fig:endpoint_orig3}
    \end{subfigure}%
    $\rightarrow$
    \begin{subfigure}{.15\columnwidth}
        \centering
        \includegraphics[width=.90\textwidth, alt={An MNIST digit}]{./images/mnist_transforms/endpoint/8-20.png}
        \caption{}
        \label{fig:endpoint_transform3}
    \end{subfigure}

    \par\medskip

    \begin{subfigure}{.15\columnwidth}
        \centering
        \includegraphics[width=.90\textwidth, alt={An MNIST digit}]{./images/mnist_transforms/raw/8-18.png}
        \caption{}
        \label{fig:endpoint_orig4}
    \end{subfigure}%
    $\rightarrow$
    \begin{subfigure}{.15\columnwidth}
        \centering
        \includegraphics[width=.90\textwidth, alt={An MNIST digit}]{./images/mnist_transforms/endpoint/8-18.png}
        \caption{}
        \label{fig:endpoint_transform4}
    \end{subfigure}

    \caption{Examples of endpoints in digits.}
    \label{fig:endpoint_examples}
\end{figure}

Property transforms, $T_7$ and $T_9$, were for the enclosed region properties.
They used a flood fill for $T_7$ and the convex hull for $T_9$. The flood fill
only fills completely enclosed regions. Flood fill is illustrated in Figure
\ref{fig:fill_examples} where \ref{fig:fill_orig1} $\rightarrow$
\ref{fig:fill_transform1} activates only the inside enclosed region of the zero.
The \ref{fig:fill_orig2} $\rightarrow$ \ref{fig:fill_transform2} and
\ref{fig:fill_orig3} $\rightarrow$ \ref{fig:fill_transform3} transformations
illustrate the effect of a break in the top of an eight and the zero. The last
transform \ref{fig:fill_orig4} $\rightarrow$ \ref{fig:fill_transform4} depicts a
digit labeled as a nine. In this example, the break in the top of the nine is
enough for that region not to be filled, while the bottom of the nine is
enclosed enough to be filled.

\begin{figure}[H]
    \centering

    \begin{subfigure}{.15\columnwidth}
        \centering
        \includegraphics[width=.90\textwidth, alt={An MNIST digit}]{./images/mnist_transforms/raw/0-1.png}
        \caption{}
        \label{fig:fill_orig1}
    \end{subfigure}%
    $\rightarrow$
    \begin{subfigure}{.15\columnwidth}
        \centering
        \includegraphics[width=.90\textwidth, alt={An MNIST digit}]{./images/mnist_transforms/fill/0-1.png}
        \caption{}
        \label{fig:fill_transform1}
    \end{subfigure}

    \par\medskip

    \begin{subfigure}{.15\columnwidth}
        \centering
        \includegraphics[width=.90\textwidth, alt={An MNIST digit}]{./images/mnist_transforms/raw/8-20.png}
        \caption{}
        \label{fig:fill_orig2}
    \end{subfigure}%
    $\rightarrow$
    \begin{subfigure}{.15\columnwidth}
        \centering
        \includegraphics[width=.90\textwidth, alt={An MNIST digit}]{./images/mnist_transforms/fill/8-20.png}
        \caption{}
        \label{fig:fill_transform2}
    \end{subfigure}%

    \par\medskip

    \begin{subfigure}{.15\columnwidth}
        \centering
        \includegraphics[width=.90\textwidth, alt={An MNIST digit}]{./images/mnist_transforms/raw/0-17.png}
        \caption{}
        \label{fig:fill_orig3}
    \end{subfigure}%
    $\rightarrow$
    \begin{subfigure}{.15\columnwidth}
        \centering
        \includegraphics[width=.90\textwidth, alt={An MNIST digit}]{./images/mnist_transforms/fill/0-17.png}
        \caption{}
        \label{fig:fill_transform3}
    \end{subfigure}

    \par\medskip

    \begin{subfigure}{.15\columnwidth}
        \centering
        \includegraphics[width=.90\textwidth, alt={An MNIST digit}]{./images/mnist_transforms/raw/9-9.png}
        \caption{}
        \label{fig:fill_orig4}
    \end{subfigure}%
    $\rightarrow$
    \begin{subfigure}{.15\columnwidth}
        \centering
        \includegraphics[width=.90\textwidth, alt={An MNIST digit}]{./images/mnist_transforms/fill/9-9.png}
        \caption{}
        \label{fig:fill_transform4}
    \end{subfigure}

    \caption{Examples of the flood fill of digits.}
    \label{fig:fill_examples}
\end{figure}

Figure \ref{fig:chull_examples} shows convex hull transform examples on the same
digits as the flood fill in Figure \ref{fig:fill_examples}. The convex hull acts
as a wrapping of the digit, capturing the entire region of the digit without
being impacted by breaks.

\begin{figure}[H]
    \centering

    \begin{subfigure}{.15\columnwidth}
        \centering
        \includegraphics[width=.90\textwidth, alt={An MNIST digit}]{./images/mnist_transforms/raw/0-1.png}
        \caption{}
        \label{fig:chull_orig1}
    \end{subfigure}%
    $\rightarrow$
    \begin{subfigure}{.15\columnwidth}
        \centering
        \includegraphics[width=.90\textwidth, alt={An MNIST digit}]{./images/mnist_transforms/chull/0-1.png}
        \caption{}
        \label{fig:chull_transform1}
    \end{subfigure}

    \par\medskip

    \begin{subfigure}{.15\columnwidth}
        \centering
        \includegraphics[width=.90\textwidth, alt={An MNIST digit}]{./images/mnist_transforms/raw/8-20.png}
        \caption{}
        \label{fig:chull_orig2}
    \end{subfigure}%
    $\rightarrow$
    \begin{subfigure}{.15\columnwidth}
        \centering
        \includegraphics[width=.90\textwidth, alt={An MNIST digit}]{./images/mnist_transforms/chull/8-20.png}
        \caption{}
        \label{fig:chull_transform2}
    \end{subfigure}%

    \par\medskip

    \begin{subfigure}{.15\columnwidth}
        \centering
        \includegraphics[width=.90\textwidth, alt={An MNIST digit}]{./images/mnist_transforms/raw/0-17.png}
        \caption{}
        \label{fig:chull_orig3}
    \end{subfigure}%
    $\rightarrow$
    \begin{subfigure}{.15\columnwidth}
        \centering
        \includegraphics[width=.90\textwidth, alt={An MNIST digit}]{./images/mnist_transforms/chull/0-17.png}
        \caption{}
        \label{fig:chull_transform3}
    \end{subfigure}

    \par\medskip

    \begin{subfigure}{.15\columnwidth}
        \centering
        \includegraphics[width=.90\textwidth, alt={An MNIST digit}]{./images/mnist_transforms/raw/9-9.png}
        \caption{}
        \label{fig:chull_orig4}
    \end{subfigure}%
    $\rightarrow$
    \begin{subfigure}{.15\columnwidth}
        \centering
        \includegraphics[width=.90\textwidth, alt={An MNIST digit}]{./images/mnist_transforms/chull/9-9.png}
        \caption{}
        \label{fig:chull_transform4}
    \end{subfigure}

    \caption{Examples of the convex hull of digits.}
    \label{fig:chull_examples}
\end{figure}

Another transform associated with the enclosed region, the flood fill of the
skeleton, was initially employed but was dropped due to poor performance. The
poor performance of the flood fill of the skeleton can be observed in Table
\ref{tab:skel_fill_metrics}, where almost no true positives for nine out of ten
MNIST classes occur.  If we also note the results for class one, $88.8\%$ of the
samples were falsely labeled as a one. It indicates that the transform was
practically unanimous in predicting all classes as the digit one. This transform
could have undergone more scrutiny, but because there was sufficient coverage in
the other transforms related to enclosed regions, it was dropped.

\begin{table}[H]
    \renewcommand{\arraystretch}{1.3}
    \caption{MNIST training metrics for the Skeleton Fill inference engine. Metric values are percentages (\%).}
    \begin{center}
    %\resizebox{\textwidth}{!}{%
    \begin{tabular}{| c | r | r | r | r | r |}
        \hline
        Class & \multicolumn{1}{|c|}{TP} & \multicolumn{1}{|c|}{TN} & \multicolumn{1}{|c|}{FP} & \multicolumn{1}{|c|}{FN} & \multicolumn{1}{|c|}{$ACC$} \\
        \hline
        \hline
        0 & 0.0 & 90.1 & 0.0 & 9.87 & 90.1 \\ \hline
        1 & 11.2 & 0.0 & 88.8 & 0.0 & 11.2 \\ \hline
        2 & 0.0 & 90.0 & 0.0 & 9.93 & 90.1 \\ \hline
        3 & 0.0 & 89.8 & 0.0 & 10.2 & 89.8 \\ \hline
        4 & 0.0 & 90.2 & 0.0 & 9.74 & 90.3 \\ \hline
        5 & 0.0 & 91.0 & 0.0 & 9.04 & 91.0 \\ \hline
        6 & 0.0 & 90.1 & 0.0 & 9.86 & 90.1 \\ \hline
        7 & 0.0 & 89.6 & 0.0 & 10.4 & 89.6 \\ \hline
        8 & 0.0 & 90.2 & 0.0 & 9.75 & 90.3 \\ \hline
        9 & 0.0 & 90.0 & 0.0 & 9.92 & 90.1 \\
        \hline
    \end{tabular}
    %}
    \end{center}
    \label{tab:skel_fill_metrics}
\end{table}

The line property transform, $T_8$, was intended to identify straight lines in
digits. It utilized the Hough Line algorithm on the skeleton. As parameters to
the Hough line, a threshold and gap of zero were used along with a minimum line
length of six pixels. Figure \ref{fig:line_examples} depicts sample line
transforms. The intermediate transform to the skeleton illustrates what is
happening in identifying lines. In the transform from \ref{fig:crossing_orig2}
$\rightarrow$ \ref{fig:line_skel2} $\rightarrow$ \ref{fig:line_transform2},
there are no lines longer than six contiguous pixels for lines to be identified
in the result. The transform from \ref{fig:line_orig3} $\rightarrow$
\ref{fig:line_skel3} $\rightarrow$ \ref{fig:line_transform3} shows three small
line segments are identified. One segment is vertical from the top right;
another segment is horizontal, descending six pixels down and left from the
vertical line, and finally, a six-pixel line extends across the bottom of the
example with a slightly positive slope.

\begin{figure}[H]
    \centering

    \begin{subfigure}{.15\columnwidth}
        \centering
        \includegraphics[width=.90\textwidth, alt={An MNIST digit}]{./images/mnist_transforms/raw/3-3.png}
        \caption{}
        \label{fig:line_orig1}
    \end{subfigure}%
    $\rightarrow$
    \begin{subfigure}{.15\columnwidth}
        \centering
        \includegraphics[width=.90\textwidth, alt={An MNIST digit}]{./images/mnist_transforms/stroke/3-3.png}
        \caption{}
        \label{fig:line_skel1}
    \end{subfigure}%
    $\rightarrow$
    \begin{subfigure}{.15\columnwidth}
        \centering
        \includegraphics[width=.90\textwidth, alt={An MNIST digit}]{./images/mnist_transforms/line/3-3.png}
        \caption{}
        \label{fig:line_transform1}
    \end{subfigure}

    \par\medskip

    \begin{subfigure}{.15\columnwidth}
        \centering
        \includegraphics[width=.90\textwidth, alt={An MNIST digit}]{./images/mnist_transforms/raw/4-5.png}
        \caption{}
        \label{fig:line_orig2}
    \end{subfigure}%
    $\rightarrow$
    \begin{subfigure}{.15\columnwidth}
        \centering
        \includegraphics[width=.90\textwidth, alt={An MNIST digit}]{./images/mnist_transforms/stroke/4-5.png}
        \caption{}
        \label{fig:line_skel2}
    \end{subfigure}%
    $\rightarrow$
    \begin{subfigure}{.15\columnwidth}
        \centering
        \includegraphics[width=.90\textwidth, alt={An MNIST digit}]{./images/mnist_transforms/line/4-5.png}
        \caption{}
        \label{fig:line_transform2}
    \end{subfigure}

    \par\medskip

    \begin{subfigure}{.15\columnwidth}
        \centering
        \includegraphics[width=.90\textwidth, alt={An MNIST digit}]{./images/mnist_transforms/raw/2-14.png}
        \caption{}
        \label{fig:line_orig3}
    \end{subfigure}%
    $\rightarrow$
    \begin{subfigure}{.15\columnwidth}
        \centering
        \includegraphics[width=.90\textwidth, alt={An MNIST digit}]{./images/mnist_transforms/stroke/2-14.png}
        \caption{}
        \label{fig:line_skel3}
    \end{subfigure}%
    $\rightarrow$
    \begin{subfigure}{.15\columnwidth}
        \centering
        \includegraphics[width=.90\textwidth, alt={An MNIST digit}]{./images/mnist_transforms/line/2-14.png}
        \caption{}
        \label{fig:line_transform3}
    \end{subfigure}

    \caption{Examples of the line transform.}
    \label{fig:line_examples}
\end{figure}

Figure \ref{fig:line_pixels_examples} shows another line example, the same four
from Figure \ref{fig:line_examples} but with varying line length parameters to
the Hough line routine. The original digit is shown again in figure
\ref{fig:line_pixel_orig1}. Figure \ref{fig:line_pixel6} shows the line
transform with a six-pixel length threshold. Figure \ref{fig:line_pixel5} shows
a five-pixel line threshold, and Figure \ref{fig:line_pixel4} shows a four-pixel
threshold.

\begin{figure}[H]
    \centering

    \begin{subfigure}{.15\columnwidth}
        \centering
        \includegraphics[width=.90\textwidth, alt={An MNIST digit}]{./images/mnist_transforms/raw/4-5.png}
        \caption{}
        \label{fig:line_pixel_orig1}
    \end{subfigure}%
    \begin{subfigure}{.15\columnwidth}
        \centering
        \includegraphics[width=.90\textwidth, alt={An MNIST digit}]{./images/mnist_transforms/line/4-5.png}
        \caption{}
        \label{fig:line_pixel6}
    \end{subfigure}%
    \begin{subfigure}{.15\columnwidth}
        \centering
        \includegraphics[width=.90\textwidth, alt={An MNIST digit}]{./images/mnist_transforms/line/4-5-5pixel.png}
        \caption{}
        \label{fig:line_pixel5}
    \end{subfigure}
    \begin{subfigure}{.15\columnwidth}
        \centering
        \includegraphics[width=.90\textwidth, alt={An MNIST digit}]{./images/mnist_transforms/line/4-5-4pixel.png}
        \caption{}
        \label{fig:line_pixel4}
    \end{subfigure}%

    \caption{Examples of the line transform with relaxed line length parameters.}
    \label{fig:line_pixels_examples}
\end{figure}

The corner property transform, $T_{10}$, used the Harris Corner detection
algorithm. This was another transform that started from the already available
skeleton. The threshold parameter used in detecting corners was $0.02$, and a
minimum distance of three pixels performed well. Figure
\ref{fig:corner_examples} illustrates examples of the corner transform
identifying corners in a four and three.

\begin{figure}[H]
    \centering

    \begin{subfigure}{.15\columnwidth}
        \centering
        \includegraphics[width=.90\textwidth, alt={An MNIST digit}]{./images/mnist_transforms/raw/4-5.png}
        \caption{}
        \label{fig:corner_orig1}
    \end{subfigure}%
    $\rightarrow$
    \begin{subfigure}{.15\columnwidth}
        \centering
        \includegraphics[width=.90\textwidth, alt={An MNIST digit}]{./images/mnist_transforms/corner/4-5.png}
        \caption{}
        \label{fig:corner_transform1}
    \end{subfigure}

    \par\medskip

    \begin{subfigure}{.15\columnwidth}
        \centering
        \includegraphics[width=.90\textwidth, alt={An MNIST digit}]{./images/mnist_transforms/raw/3-18.png}
        \caption{}
        \label{fig:corner_orig2}
    \end{subfigure}%
    $\rightarrow$
    \begin{subfigure}{.15\columnwidth}
        \centering
        \includegraphics[width=.90\textwidth, alt={An MNIST digit}]{./images/mnist_transforms/corner/3-18.png}
        \caption{}
        \label{fig:corner_transform2}
    \end{subfigure}

    \caption{Examples of the corner transform.}
    \label{fig:corner_examples}
\end{figure}

\subsection{Transform Training Data}

A Python program,
\href{https://gitlab.com/paulwhitten/2023-svm-xai/-/blob/main/transform_parallel.py}{\texttt{transform\_parallel.py}}\cite{whitten2024icmi_source},
was implemented to transform the training data.  The program will take an MNIST
database as input and output a new transformed MNIST database for each of the
eleven transforms. Initially, the program performed transforms in a serial
fashion using one core. Often, the transformation would take over an hour to
complete on the 60,000-element training set using an Intel i9-10900 CPU @2.80GHz
with 96GB RAM. The program would run in under seven minutes when it was
refactored to exploit parallelism.

\subsection{Train ML Models}

Several ML algorithms were utilized in training models as inference engines for
the property transformations. Multi-layer perceptrons (MLP) and support vector
machines (SVM) from the scikit-learn\cite{scikitlearn} python package were used,
as well as convolutional neural networks and ResNet deep learning algorithms
from TensorFlow\cite{tensorflow2015-whitepaper}.

\begin{figure}[H]
    \centerline{\includegraphics[width=11cm, alt={A diagram of the property transforms and inference engines for handwritten characters}]{./images/property-ml-handwriting-dot.png}}
    \caption{Property based explainable transforms and inference engines for handwriting recognition.}
    \label{fig:prop_trans_inference_handwriting}
\end{figure}

The MLP implementation of training the inference engines is implemented in
\href{https://gitlab.com/paulwhitten/2023-svm-xai/-/blob/main/train_nn_transforms.py}{\texttt{train\_nn\_transforms.py}}\cite{whitten2024icmi_source}.
Three hidden layers of 128, 256, and 128 perceptrons with the rectified linear
unit activation function gave good results.

The SVM inference engines used the support vector classifier with the radial
basis function kernel. The implementation of the training routine is in the file
\href{https://gitlab.com/paulwhitten/2023-svm-xai/-/blob/main/train_svm_transforms.py}{\texttt{train\_svm\_transforms.py}}\cite{whitten2024icmi_source}.
SVM inference engines took much longer to train than the MLP inference engines.

After training the property transforms and inference engines, a system may be
constructed resembling Figure~\ref{fig:prop_trans_inference_handwriting}.

\subsection{Knowledgebase}

The knowledgebase stores data for determining effectiveness so global decisions
can be made based on the potentially conflicting opinions of the PDFs. A series
of Python programs were used to construct the knowledgebase. Python utility
functions also implement functionality to access the data in the knowledgebase
at runtime.

The programs construct and then serialize the knowledgebase to persistent
storage. Persistent storage used for the handwriting knowledgebases is comprised
of a JSON file. A database or other storage mechanism may be employed for larger
datasets. In addition to storing the confusion matrix for each PDF, the various
effectiveness metrics were calculated in advance and serialized in the
knowledgebase. Rather than store redundant information, space may be saved by
merely saving the optimal effectiveness metric for each PDF. Because this work
was research-focused, much more was stored in the knowledgebase than was
necessary for a typical application.

In calculating the metrics, a PDF's overall metric is insufficient. The metric
for each class using a one-versus-others scheme must be calculated and stored.
Because of the one-versus-others scheme, a significant imbalance is introduced.
It is crucial to select a metric that is resilient to data imbalance. The
metrics that were calculated and stored in the knowledgebase included:

\begin{itemize}
    \item Accuracy
    \item Precision
    \item Recall 
    \item Specificity 
    \item Area under the ROC curve
    \item Matthew's correlation coefficient
    \item Cohen's Kappa
    \item $E_{PARS}$
\end{itemize}

% TODO In addition to storing metrics

\subsection{Voting}

The purpose of the voting scheme is to decide among the potentially conflicting
opinions from the explainable property transformation inferences. Information
from the knowledgebase and explainable property inferences are input to the
voting scheme. Probabilistic and ML voting schemes were employed and compared.
The probabilistic voter had a high degree of explainability, while the ML voter
occasionally provided results that did not correspond to opinions related to
explainable properties.

In the probabilistic voting scheme, the knowledgebase retrieves effectiveness
metrics for each inference engine based on its opinion. For example, if the
stroke IE suggested a two, the effectiveness of the stroke inference engine at
identifying a two was retrieved and used to weight the effectiveness of the
digit two. Recall that the weighted effectiveness of a digit, $c$, is $WE(c)$ in
\eqref{eq:weighted_effectiveness} is given as the sum of the effectiveness for
the inference engines that voted for $c$. Weighted effectiveness is then
calculated for each class that an inference engine suggested in the same
fashion.

After calculating the weighted effectiveness of each digit suggested, the
weighted effectiveness of all digits are summed as the total weighted
effectiveness of all classes. Confidence for each suggested digit is given by
the weighted effectiveness of that digit over the total weighted effectiveness
of all digits. The digit with the highest confidence wins.

Various effectiveness measures were utilized. There are ten classes in MNIST,
one for each decimal digit. When taking effectiveness in a one versus others
strategy, there will be a high imbalance factor of nine due to the one-to-others
ratio of 1 to 9. It was important to select effectiveness metrics that were
resilient to data imbalance. Metrics without such imbalance resiliency, such as
accuracy and recall, often gave misleading metrics due to the data imbalance.

\subsection{Explanation}

The explanation routine takes the digits suggested by the voter and the
confidence of those opinions and organizes them in descending order of
confidence. The winning digit with the highest confidence is posed as the
solution with alternatives and their relative confidence.

A textual explanation is next constructed for the winning decision and each
alternative. The explanation is based on the opinions of the various inference
engines suggesting a particular digit. The explanation routine orders the
winning decision and alternatives in descending order based on effectiveness.
The explanation routine composes a textual rationale for each class that was
voted on, linking the explainable properties to the suggestion.

Finally, explainability metrics of the winning decision and alternatives are
presented to the user. If unexplainable components are used in the system, the
explainability for the digits suggested by the unexplainable components are
calculated from \eqref{eq:explainability}. Digits without votes from an
unexplainable component will have full explainability.

\section{Explainable Property-Based Methodology and EMNIST}

The EMNIST dataset is split in six different ways, based on classes included and
the balancing of classes in each split. The split primarily used in this work is
the EMNIST balanced data set which contains 131,600 samples in 46 classes.

%consisting of decimal digits, 26 uppercase letters, and 10 lowercase letters.

%While perusing samples in the EMNIST the dataset there were observations of
%ambiguous and mislabeled samples.

%Some of the label problems observed in EMNIST, from a cursory review, are shown in
%Figure~\ref{fig:emnist_label_errors}. Each row in the figure represents samples from
%a class.  The first sample of each row appears to be acceptable,
%subsequent samples in each row are not. One particular class, F, had a large
%percentage, over 30\%, observed as lowercase f. Other classes were noted to have
%ambiguous or overlapping symbols such as U and V; $0$, O, and D; and L, I, l and
%the digit 1.

The high incidence of label errors and ambiguous data causes accuracy problems
with ML and especially exacerbates XAI system results. There was a strong desire
to address some of the issues.

In an attempt to improve explainable results, the removal of ambiguous and
mislabeled data was explored. Data and pruning of the training and test sets
were performed in several ways and compared.

The first pruning scheme was based on a threshold of 75\% confidence, which
eliminated 5\% of the training set. Results of threshold pruning were observed
on a per-class basis, and it was noted in Table \ref{tab:threshold_pruning_qty}
that in some classes, over half of the samples were removed.  E.g., the I and L
classes had more than 1200 of the 2400 samples for each class pruned. This
pruning caused a highly unbalanced training set and resulted in only 13\%
accuracy on uppercase letters. Higher thresholds up to 95\% were also attempted.
However, poor accuracy results were observed\cite{whitten23}.

\begin{table}
    \renewcommand{\arraystretch}{1.3}
    %\centering
    \caption{The number of pruned samples based on a 75\% confidence threshold}
    \begin{center}
    %\resizebox{\textwidth}{!}
\end{center}
    \label{tab:threshold_pruning_qty}
\end{table}

The second pruning scheme limited the pruning to only the worst 5\% confidence
of each class. Wrong classifications were especially penalized by assigning the
confidence a negative value via multiplication by $-1.0$. This scheme used a
K-fold Cross-validation to ensure that confidence of the training set samples
were obtained out of sample and without bias.

Another means of pruning involved leveraging Cleanlab to remove issues from the
data. Cleanlab was also used iteratively with three passes, and results are
reviewed in Chapter \ref{ch:handwriting_results}.

\section{Case-Based Explainable Method and MNIST}

This section outlines the implementation of the case-based architecture for
handwritten digits. Recall from the case-based method that the training index
can be constructed to store calculated distances to speed calculations. Because
each MNIST sample has $28\times28=784$ pixels, each with a range of $[0, 255]$,
each sample has an astronomical number of possibilities. The training set also
has 60,000 samples, so there was concern that the algorithm for finding KNN
using distance would be costly. With the high number of pixels in each sample
and considering that each pixel can take 256 values, there is a low likelihood
of collisions in training samples. Few collisions present few opportunities to
skip calculations compared to a low-dimensionality dataset.

Due to these performance concerns, the case-based method was applied to
unchanged MNIST and MNIST, with principal component analysis applied to compare
performance. Using PCA to reduce the dimensionality of the dataset had
advantages in accuracy, compute time, and explainability.

 %TODO: can we add memory to the list of PCA benefits as well?

\subsection{Data Processing}

No additional data processing was required for the case-based method on MNIST.
When using the reduced MNIST dataset, PCA from scikit-learn\cite{scikitlearn}
was used. An analysis of the MNIST data showed that about 87 principal
components would capture $90\%$ of the variance in the data. Figure
\ref{fig:pca_variance} shows a plot of the cumulative variance to the number of
principal components of the MNIST dataset.

\begin{figure}[h]
    \centerline{\includegraphics[width=15cm, alt={Cumulative variance for MNIST based on the number of principal components}]{./images/pca_variance.png}}
    \caption{A plot of the number of principal components to the cumulative variance in MNIST data.}
    \label{fig:pca_variance}
\end{figure}

Various numbers of principal components were tried and compared for performance.
The number of principal components that performed best (speed, explainability,
and accuracy) was around 75.

\subsection{Train ML Model}

A SVM with a radial basis function kernel was used in the case-based method. The
SVM performed identically to the SVM used in the unexplainable inference engine
of the property-based method on unchanged MNIST. Using PCA, better performance
was observed. Particular results are in Section
\ref{sec:case_based_handwriting_results}. It took about 2 minutes to train an
SVM on the complete MNIST training data and only 20 seconds to train an SVM with
the MNIST data reduced to 75 principal components.

\subsection{Training Index}

As suspected, there were no collisions in the MNIST data. As a result, caching
distance calculations in the training index was not beneficial. The resulting
training index for MNIST was the base MNIST image data and labels. A lookup
could be performed based on the index in the MNIST database.

In the case of MNIST reduced using PCA, the PCA transforms of the training data,
and the principal components were also stored to provide the result. Figure
\ref{fig:mnist_eigendigits} presents the first 81 principal components or
eigendigits.

\begin{figure}[H]
    \centerline{\includegraphics[width=15cm, alt={MNIST eigendigits from PCA}]{./images/mnist-81-eigendigits.png}}
    \caption{A visualization of the first 81 principal components (eigendigits) from MNIST.}
    \label{fig:mnist_eigendigits}
\end{figure}


\subsection{Query Scheme}

The simple numeric index into the training set was used as lookup for the MNIST
and PCA data. When presented with a sample, the case based explainable
architecture iterates over the training set and calculates the distances to all
of the training samples. In order to find the K nearest training samples, the
first K elements of the sorted list of distances are retrieved.

The time it takes the system to calculate distances to training samples and find
the K nearest neighbors for MNIST averages 360 milliseconds in a Python program
on an Intel i9-10900 CPU @2.80GHz. Using a reduced MNIST with 75 principal
components results in 290 milliseconds to find K nearest neighbors using the
same hardware.

\subsection{Case-Based Explanation Routine}

From the distance calculation for the K nearest neighbors, weight and
correspondence are found for the classes in the K neighbors. The explainable
case-based architecture presents the SVM inference prediction and the neighbors'
correspondence figures. The system indicates whether neighbor correspondence
agrees or conflicts with the SVM prediction. Information about the neighbors in
the form of indices from the training set is also provided. From these indices,
the actual MNIST and, if applicable, PCA data can be provided. Explainable
examples are presented in Section \ref{sec:case_based_explainable_examples}.
